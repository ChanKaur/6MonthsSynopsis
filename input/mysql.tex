\image{0.6}{images/mysql.png}{MySQL Logo}
\hspace{-1.8em} {\bf What is MySQL?}\\\\
MySQL is a relational database management system (RDBMS) that runs as a server providing multi-user access to a number of databases. It is named after developer "Michael Widenius" daughter, My. The SQL phrase stands for Structured Query Language.\\\\
The MySQL development project has made its source code available under the terms of the GNU General Public License, as well as under a variety of proprietary agreements. MySQL was owned and sponsored by a single for-profit firm, the Swedish company MySQL AB, now owned by Oracle Corporation.\\\\
Free-software-open source projects that require a full-featured database management system often use MySQL. For commercial use, several paid editions are available, and offer additional functionality. Applications which use MySQL databases include: Joomla, WordPress, MyBB, phpBB, Drupal and other software built on the LAMP software stack. MySQL is also used in many high-profile, large-scale World Wide Web products, including Wikipedia, Google (though not for searches) and Facebook. The data in MySQL is stored in database objects called tables.\\\\
A table is a collection of related data entries and it consists of columns and rows. Databases are useful when storing information categorically. A company may have a
database with the following tables: ”Employees”, ”Products”, ”Customers” and ”Orders”.\\\\
\subsection{Queries}
A query is a question or a request.\\
With MySQL, we can query a database for specific information and have a recordset returned.\\\\
\subsection{Create a Connection to a MySQL Database}
Before you can access data in a database, you must create a connection to the database.\\
In PHP, this is done with the mysql connect() function. Syntax mysql connect(servername,username)\\
Example:\\
In the following example we store the connection in a variable (\$con) for later use in
the script. The ”die” part will be executed if the connection fails:\\
\begin{verbatim}
<?php
$con = mysql_connect("localhost","username","password");
if (!$con)
{
die(’Could not connect: ’ . mysql_error());
}
// some code
?>
\end{verbatim}
\subsection{Closing a Connection}
The connection will be closed automatically when the script ends. To close the connection before, use the mysql close() function:\\\\
\begin{verbatim}
<?php
$con = mysql_connect("localhost","username","password");
if (!$con)
{
die(’Could not connect: ’ . mysql_error());
}
// some code
mysql_close($con);
?>
\end{verbatim}
